\title{Two electron integrals - Particle in a cube}
\author{ Jellium Team
}
\date{\today}

\documentclass[12pt]{article}

\begin{document}
\maketitle

\begin{abstract}
Analytical expressions for two-electron repulsion integrals in particle
in a cube basis
\end{abstract}

\section{Introduction}
Two-electron integrals for particle in a cube:

Each integral is of the form:
\begin{equation}
(ab|cd) = \int \frac{ \phi_a(r_1) \phi_b(r_1) \phi_c(r_2) \phi_d(r_2)}{|r_1 - r_2|} dr_1 dr_2
\end{equation}

where $\phi_a(r_1) =\left(\frac{2}{\pi}\right)^{3/2} {\rm sin}(a_x x_1) {\rm sin}(a_y y_1) {\rm sin}(a_z z_1)$.

Using the trigonometric identity $2 {\rm sin}(a_x x_1) {\rm sin}(b_x x_1) = {\rm cos}((a_x - b_x) x_1) - {\rm cos}((a_x + b_x) x_1)$ leads to the realization that the two electron integrals $(ab|cd)$ can be expanded as linear combinations of integrals of the form 
\begin{equation}
(p|q) = \frac{1}{\pi^6} \int \frac{{\rm cos}(p_x x_1) {\rm cos}(p_y y_1) {\rm cos}(p_z z_1)  {\rm cos}(q_x x_2) {\rm cos}(q_y y_2) {\rm cos}(q_z z_2) }{|r_1 - r_2|} dr_1 dr_2.
\end{equation}


%\bibliographystyle{abbrv}
%\bibliography{main}

\end{document}
